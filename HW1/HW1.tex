\documentclass[11pt,a4paper]{ctexart}
%以下为所使用的宏包
\usepackage{ulem}%下划线
\usepackage{amsmath,amsfonts,amssymb,amsthm,amsbsy}%数学符号
\usepackage{graphicx}%插入图片
\usepackage{booktabs}%三线表
%\usepackage{indentfirst}%首行缩进
\usepackage{tikz}%作图
\usepackage{appendix}%附录
\usepackage{array}%多行公式/数组
\usepackage{makecell}%表格缩并
\usepackage{siunitx}%SI单位--\SI{number}{unit}
\usepackage{mathrsfs}%数学字体
\usepackage{enumitem}%列表间距
\usepackage{multirow}%列表横向合并单元格
\usepackage[colorlinks,linkcolor=red,anchorcolor=blue,citecolor=green]{hyperref}%超链接引用
\usepackage{float}%图片、表格位置排版
\usepackage{pict2e,keyval,fp,diagbox}%带有斜线的表格
\usepackage{fancyvrb,listings}%设置代码插入环境
\usepackage{minted}%代码环境设置
\usepackage{fontspec}%字体设置
\usepackage{color,xcolor}%颜色设置
\usepackage{titlesec} %自定义标题格式
\usepackage{tabularx}%列表扩展
\usepackage{authblk}%titlepage作者信息
\usepackage{nicematrix}%更好的矩阵标定
\usepackage{fbox}%更多浮动体盒子



%以下是页边距设置
\usepackage[left=0.5in,right=0.5in,top=0.81in,bottom=0.8in]{geometry}

%以下是段行设置
\linespread{1.4}%行距
\setlength{\parskip}{0.1\baselineskip}%段距
\setlength{\parindent}{2em}%缩进


%其他设置
\numberwithin{equation}{section}%公式按照章节编号
\newenvironment{point}{\raggedright$\blacktriangleright$}{}
\newenvironment{algorithm}[1]{\vspace{12pt} \hrule\hrule \vspace{3pt} \noindent\textbf{\color[HTML]{E63F00}Algorithm } \,\textit{#1} \vspace{3pt} \hrule\vspace{6pt}}{\vspace{6pt}\hrule\hrule \vspace{12pt}} % 算法伪代码格式环境


%代码环境\lst设置
\definecolor{CodeBlue}{HTML}{268BD2}
\definecolor{CodeBlue2}{HTML}{0000CD}
\definecolor{CodeGreen}{HTML}{2AA1A2}
\definecolor{CodeRed}{HTML}{CB4B16}
\definecolor{CodeYellow}{HTML}{B58900}
\definecolor{CodePurPle}{HTML}{D33682}
\definecolor{CodeGreen2}{HTML}{859900}
\lstset{
    basicstyle=\tt,%字体设置
    numbers=left, %设置行号位置
    numberstyle=\tiny\color{black}, %设置行号大小
    keywordstyle=\color{black}, %设置关键字颜色
    stringstyle=\color{CodeRed}, %设置字符串颜色
    commentstyle=\color{CodeGreen}, %设置注释颜色
    frame=single, %设置边框格式
    escapeinside=`, %逃逸字符(1左面的键),用于显示中文
    %breaklines, %自动折行
    extendedchars=false, %解决代码跨页时,章节标题,页眉等汉字不显示的问题
    xleftmargin=2em,xrightmargin=2em, aboveskip=1em, %设置边距
    tabsize=4, %设置tab空格数
    showspaces=false, %不显示空格
    emph={TRUE,FALSE,NULL,NAN,NA,<-,},emphstyle=\color{CodeBlue2}, %其他高亮}
}


%节标题格式设置
\titleformat{\section}[block]{\large\bfseries}{Exercise \arabic{section}}{1em}{}[]
\titleformat{\subsection}[block]{}{    \arabic{section}.(\alph{subsection})}{1em}{}[]
% \titleformat{\subsubsection}[block]{\normalsize\bfseries}{    \arabic{subsection}-\alph{subsubsection}}{1em}{}[]
% \titleformat{\paragraph}[block]{\small\bfseries}{[\arabic{paragraph}]}{1em}{}[]


% \titleformat{\sectioncommand}[shape]{format}{title-label}{sep}{before-title}[after-title]



% 中文字号
% 初号42pt, 小初36pt, 一号26pt, 小一24pt, 二号22pt, 小二18pt, 三号16pt, 小三15pt, 四号14pt, 小四12pt, 五号10.5pt, 小五9pt


\begin{document}

\begin{center}\thispagestyle{plain}

{\LARGE\textbf{STAT 430-2 2025 Winter}}

{\Large\textbf{HW1}}

Tuorui Peng\footnote{TuoruiPeng2028@u.northwestern.edu}
\end{center}

\thispagestyle{myheadings}\markright{Compiled using \LaTeX}
\pagestyle{myheadings}\markright{Tuorui Peng}


% 4.2.13, 4.2.14, 4.2.15, 4.2.16, 4.2.17, 4.2.22, 4.2.23, 4.3.13((a),(b)), 4.4.8, 4.4.10.

\section{4.2.13}

Note that we can write:
\begin{align*}
    kX\sim \mathrm{ Unif }(0,k) \mathop{ = }\limits^{\mathrm{ d } } \mathrm{ Unif }\{0,1,\ldots, k-1\} +  \mathrm{ Unif }(0,1) := N + U\quad N\perp \!\!\!\perp U
\end{align*}
And we have
\begin{align*}
    \mathbb{E}\left[ X|Y \right] =& \dfrac{ 1 }{ k }\mathbb{E}\left[ kX| kX-[kX] \right]\\
    =& \dfrac{ 1 }{ k }\mathbb{E}\left[ N+U | U \right]\\
    =& \dfrac{ 1 }{ k }(U + \mathbb{E}\left[ N \right] )\\
    =& \dfrac{ 1 }{ k }( Y + \dfrac{ k-1 }{ 2 })\\
\end{align*}


\section{4.2.14}
We can write
\begin{align*}
    \mathbb{E}\left[ X|Y \right] =& \mathbb{E}\left[ X\mathbf{1}_{X\leq t} + X\mathbf{1}_{X<t} | \max (X,t) \right]\\
    =& \mathbb{E}\left[ X\mathbf{1}_{\max (X,t) = t} + X\mathbf{1}_{\max (X,t) = X} | \max (X,t) \right]\\
    =& \mathbf{1}_{Y = t}\mathbb{E}\left[ X|X\leq t \right] + Y\mathbf{1}_{Y > t}
\end{align*}

The other side of $ Z=\min (X,t) $ is symmetric to the above, so we have
\begin{align*}
    \mathbb{E}\left[ X|Z \right] =& \mathbf{1}_{Z = t}\mathbb{E}\left[ X|X\geq t \right] + Z\mathbf{1}_{Z < t} 
\end{align*}

\section{4.2.15}

\subsection{}
We can directly write
\begin{align*}
    Z|\theta =1 =& [X,Y] \mathop{ = }\limits^{\mathrm{ d } } P\times P\\
    Z|\theta =0 =& [Y,X] \mathop{ = }\limits^{\mathrm{ d } } P\times P  
\end{align*}
where $ X,Y \mathop{ \sim }\limits^{i.i.d.}  P $. i.e. $ Z \perp \!\!\!\perp \theta $.


\subsection{}

We have
\begin{align*}
    \mathbb{E}\left[ g(X,Y)  |Z\right]  =& \mathbb{E}\left[ \mathbb{E}\left[ g(X,Y) |\theta ,Z\right] |\theta  \right]  \\
    =& \mathbb{P}\left( \theta =1  \right) \mathbb{E}\left[ g(X,Y)|\theta =1,Z \right] + \mathbb{P}\left( \theta =0  \right) \mathbb{E}\left[ g(X,Y)|\theta =0,Z \right]\\
    =& pg(Z_1,Z_2) + (1-p)g(Z_2,Z_1)
\end{align*}


\section{4.2.16}

\subsection{}
Note that from the property of conditional variance:
\begin{align*}
    \mathbb{E}\left[ X|\mathcal{G} \right] = \mathop{ \arg\min }\limits_{g\in \mathcal{G} } \mathbb{E}\left[ (X-g)^2 \right]   
\end{align*}
Then we have for $ \mathcal{G}_1\subseteq \mathcal{G}_2 $:
\begin{align*}
    \mathbb{E}\left[ var(X|\mathcal{G}_2) \right] =& \mathbb{E}\left[ \mathop{ \arg\min }\limits_{g\in \mathcal{G}_2 } \mathbb{E}\left[ (X-g)^2 \right]   \right]   \\
    \leq&  \mathbb{E}\left[ \mathop{ \arg\min }\limits_{g\in \mathcal{G}_1 } \mathbb{E}\left[ (X-g)^2 \right]   \right] \\
    =& \mathbb{E}\left[ var(X|\mathcal{G}_1) \right]
\end{align*}

\subsection{}
We have
\begin{align*}
    var(X)=& \mathbb{E}\left[ \mathbb{E}\left[ (X-\mathbb{E}\left[ X \right] )^2 | \mathcal{G} \right]  \right]  \\
    =&\mathbb{E}\left[ \mathbb{E}\left[ (X-\mathbb{E}\left[ X|\mathcal{G} \right] + \mathbb{E}\left[ X|\mathcal{G} \right] -\mathbb{E}\left[ X \right]  )^2 | \mathcal{G} \right]  \right]\\
    =&\mathbb{E}\left[ \mathbb{E}\left[ (X-\mathbb{E}\left[ X|\mathcal{G} \right] )^2   |\mathcal{G} \right]  \right]  + \mathbb{E}\left[ (\mathbb{E}\left[ X|\mathcal{G} \right] -\mathbb{E}\left[ X \right]  )^2 |\mathcal{G} \right]   \\
    =& \mathbb{E}\left[ var(X|\mathcal{G}) \right] + var(\mathbb{E}\left[ X|\mathcal{G} \right] )
\end{align*}


\section{4.2.17}

\subsection{}
We have
\begin{align*}
    \mathbb{E}\left[ X \right] =& \mathbb{E}\left[ \mathbb{E}\left[ \sum_{i=1}^N \xi _i |N \right]  \right]\\
    =& \sum_{j=1}^\infty \mathbb{P}\left( N=j \right) \sum_{i=1}^j \mathbb{E}\left[ X_i \right] \\
    =& \sum_{i=1}^\infty\sum_{j=i}^\infty \mathbb{P}\left( N= j \right) \mathbb{E}\left[ \xi _i \right] \\
    =& \sum_{i=1}^\infty \mathbb{E}\left[ \xi _i \right] \sum_{j=i}^\infty \mathbb{P}\left( N= j \right) <\infty
\end{align*}

\subsection{}
We have
\begin{align*}
    \mathbb{E}\left[ X^2 \right] =& \mathbb{E}\left[ \mathbb{E}\left[ \left( \sum_{i=1}^N \xi _i \right)^2 |N \right]  \right]\\
    =& \sum_{j=1}^\infty \mathbb{P}\left( N=j \right)  \big(j\cdot var(\xi _i) + 2j^2 \mathbb{E}\left[ \xi _i \right] ^2\big)\\
    =& \sum_{i=1}^\infty\sum_{j=i}^\infty \mathbb{P}\left( N=j  \right) \big(var(\xi _i) + 2j\mathbb{E}\left[ \xi _i \right] ^2\big) \\
    =& var(\xi _i)\mathbb{E}\left[ N \right] + \mathbb{E}\left[ \xi _i \right]^2 \mathbb{E}\left[ N^2 \right]  <\infty\\
     \Rightarrow var(X)=& \mathbb{E}\left[ X^2 \right] - \mathbb{E}\left[ X \right] ^2\\
     =& var(\xi _i)\mathbb{E}\left[ N \right] + \mathbb{E}\left[ \xi _i \right]^2 \mathbb{E}\left[ N^2 \right] - \mathbb{E}\left[ \xi _i \right]^2 \mathbb{E}\left[ N \right]^2\\
     =&   var(\xi _i)\mathbb{E}\left[ N \right] + \mathbb{E}\left[ \xi _i \right]^2var(N)
\end{align*}

\section{4.2.22}

\subsection{}
We have for $ \mathcal{G}\subseteq\mathcal{F} $:
\begin{align*}
    \mathbb{E}\left[ \left\vert X \right\vert ^p | \mathcal{G} \right]  =& \int \left\vert x \right\vert ^p  \,\mathrm{d}\mathbb{P}\left( \left\vert X \right\vert |\mathcal{G} \right) \\
    \mathop{ = }\limits^{\text{integration by parts}} & \int p\left\vert x \right\vert ^{p-1} \mathbb{P}\left( \left\vert X \right\vert >x|\mathcal{G} \right) \,\mathrm{d}x
\end{align*}

\subsection{}
Further we have
\begin{align*}
    \mathbb{E}\left[ \left\vert X \right\vert ^p | \mathcal{G} \right]  =&\int p\left\vert x \right\vert ^{p-1} \mathbb{P}\left( \left\vert X \right\vert >x|\mathcal{G} \right) \,\mathrm{d}x\\
    =& \int_0^a +\int_a^\infty p\left\vert x \right\vert ^{p-1} \mathbb{P}\left( \left\vert X \right\vert >x|\mathcal{G} \right) \,\mathrm{d}x\\
    \geq & \int_0^a p\left\vert x \right\vert ^{p-1} \mathbb{P}\left( \left\vert X \right\vert >a|\mathcal{G} \right) \,\mathrm{d}x\\
    =& \mathbb{P}\left( \left\vert X \right\vert >a|\mathcal{G} \right) a^{p}\\
     \Rightarrow \mathbb{P}\left( \left\vert X \right\vert >a|\mathcal{G} \right) \leq& a^{-p}\mathbb{E}\left[ \left\vert X \right\vert ^p | \mathcal{G} \right]
\end{align*}



\section{4.2.23}

Using similar argument as Proposition 1.3.17 in textbook, we have by Cauchy-Schwarz inequality:
\begin{align*}
     1=\dfrac{ 1 }{ p }+\dfrac{ 1 }{ q }=& \dfrac{ \left\Vert X \right\Vert _p ^p }{ p\left\Vert X \right\Vert _p ^p } + \dfrac{ \left\Vert Y \right\Vert _q ^q }{ q\left\Vert X \right\Vert _q ^q }\\
     \geq & \dfrac{ \mathbb{E}\left[ \left\vert XY \right\vert  \right]  }{ \left\Vert X \right\Vert _p \left\Vert Y \right\Vert _q }\\
      \Rightarrow \mathbb{E}\left[ \left\vert XY \right\vert  \right]  \leq& \left\Vert X \right\Vert _p \left\Vert Y \right\Vert _q   
\end{align*}

\section{4.3.13}

\subsection{}
As we stated previously in 4.2.16, for any given $ N $, 
\begin{align*}
    \mathbb{E}\left[ (\mathbb{E}\left[ X| \mathcal{G}_m \right] -\mathbb{E}\left[ X| \mathcal{G}_N \right] )^2 \right] ,\quad m\geq N
\end{align*}
is a increasing function in $ m $ (because $ \mathbb{E}\left[ (X-\mathbb{E}\left[ X| \mathcal{G}_m \right])^2 \right]  $ is decreasing). On the other hand since $ X\in L^2(\mathcal{F}) $, this quantity is bounded. The above means that $ \mathbb{E}\left[ (\mathbb{E}\left[ X| \mathcal{G}_m \right] -\mathbb{E}\left[ X| \mathcal{G}_N \right] )^2 \right] $ has some limit, i.e.
\begin{align*}
    \mathbb{E}\left[ (\mathbb{E}\left[ X| \mathcal{G}_m \right] -\mathbb{E}\left[ X| \mathcal{G}_n \right] )^2 \right] \to 0,\quad m,n\to \infty
\end{align*}
i.e. it's a cauchy sequence. 

\subsection{}

For Hilbert space $ L^2(\mathcal{G}) $, we see that $ \exists h $ s.t. the cauchy sequence $ \mathbb{E}\left[ X| \mathcal{G}_n \right]  $ converges to $ h $ in $ L^2(\mathcal{F}) $. And by the Orthogonal Projection Theorem, we have this $ h $ being the unique minimizer that reaches $ \min_{g\in \mathcal{G}}\mathbb{E}\left[ (X-g)^2|\mathcal{G} \right]  $. i.e. 
\begin{align*}
     \mathbb{E}\left[ X| \mathcal{G}_m \right] \to h = \mathbb{E}\left[ X| \mathcal{G} \right]
\end{align*}


\section{4.4.8}

\subsection{}
We have:
\begin{align*}
    \mathbb{E}\left[ \xi _1|S_n \right] = \ldots = \mathbb{E}\left[ \xi _n|S_n \right] 
\end{align*}
thus
\begin{align*}
    \mathbb{E}\left[ \xi _1|S_n \right] =& n^{-1}\mathbb{E}\left[ \sum_{i=1}^n \xi _i | S_n \right] = n^{-1}S_n 
\end{align*}

\subsection{}
Consider transformation $ (\xi _1,\xi _2)\mapsto (\xi _1, u)=(\xi _1,\xi _1+\xi _2) $, which gives
\begin{align*}
    f_{\xi _1,u}(\xi _1,u)=& \lambda ^2e^{-\lambda u}\mathbb{1}_{ \xi _1>0,u>\xi _1 }\\
    f_u(u)=&\lambda ^2 u e^{-\lambda u} \mathbb{1}_{ u>0 }\\ 
\end{align*}
And we have
\begin{align*}
    f(\xi _1|u)=&\dfrac{ 1 }{ u } \mathbf{1}_{ \xi _1>0,u>\xi _1 }\sim \mathrm{ Unif }(0,u)  
\end{align*}
i.e.
\begin{align*}
    \mathbb{P}\left( \xi _1\leq b|S_2 \right) =& \dfrac{ b }{ S_2 }\mathbf{1}_{ 0<b<S_2 } 
\end{align*}

\section{4.4.10}

Note that we have
\begin{align*}
    cov(X-\dfrac{ \mathbb{E}\left[ XY \right]  }{ \mathbb{E}\left[ Y^2 \right]  }Y ,Y) =0 
\end{align*}
and both $ X-\dfrac{ \mathbb{E}\left[ XY \right]  }{ \mathbb{E}\left[ Y^2 \right]  }Y $ and $ Y $ are gaussian, thus are independent. Now we have
\begin{align*}
    \mathbb{E}\left[ X|Y \right] =&\mathbb{E}\left[ X-\dfrac{ \mathbb{E}\left[ XY \right]  }{ \mathbb{E}\left[ Y^2 \right]  }Y + \dfrac{ \mathbb{E}\left[ XY \right]  }{ \mathbb{E}\left[ Y^2 \right]  }Y |Y  \right]  \\
    =& 0 + \dfrac{ \mathbb{E}\left[ XY \right]  }{ \mathbb{E}\left[ Y^2 \right]  }Y= \rho Y                                                 
\end{align*}




















































\end{document}