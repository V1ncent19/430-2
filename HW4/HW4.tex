\documentclass[11pt,a4paper]{ctexart}
%以下为所使用的宏包
\usepackage{ulem}%下划线
\usepackage{amsmath,amsfonts,amssymb,amsthm,amsbsy}%数学符号
\usepackage{graphicx}%插入图片
\usepackage{booktabs}%三线表
%\usepackage{indentfirst}%首行缩进
\usepackage{tikz}%作图
\usepackage{appendix}%附录
\usepackage{array}%多行公式/数组
\usepackage{makecell}%表格缩并
\usepackage{siunitx}%SI单位--\SI{number}{unit}
\usepackage{mathrsfs}%数学字体
\usepackage{enumitem}%列表间距
\usepackage{multirow}%列表横向合并单元格
\usepackage[colorlinks,linkcolor=red,anchorcolor=blue,citecolor=green]{hyperref}%超链接引用
\usepackage{float}%图片、表格位置排版
\usepackage{pict2e,keyval,fp,diagbox}%带有斜线的表格
\usepackage{fancyvrb,listings}%设置代码插入环境
\usepackage{minted}%代码环境设置
\usepackage{fontspec}%字体设置
\usepackage{color,xcolor}%颜色设置
\usepackage{titlesec} %自定义标题格式
\usepackage{tabularx}%列表扩展
\usepackage{authblk}%titlepage作者信息
\usepackage{nicematrix}%更好的矩阵标定
\usepackage{fbox}%更多浮动体盒子



%以下是页边距设置
\usepackage[left=0.5in,right=0.5in,top=0.81in,bottom=0.8in]{geometry}

%以下是段行设置
\linespread{1.4}%行距
\setlength{\parskip}{0.1\baselineskip}%段距
\setlength{\parindent}{2em}%缩进


%其他设置
\numberwithin{equation}{section}%公式按照章节编号
\newenvironment{point}{\raggedright$\blacktriangleright$}{}
\newenvironment{algorithm}[1]{\vspace{12pt} \hrule\hrule \vspace{3pt} \noindent\textbf{\color[HTML]{E63F00}Algorithm } \,\textit{#1} \vspace{3pt} \hrule\vspace{6pt}}{\vspace{6pt}\hrule\hrule \vspace{12pt}} % 算法伪代码格式环境


%代码环境\lst设置
\definecolor{CodeBlue}{HTML}{268BD2}
\definecolor{CodeBlue2}{HTML}{0000CD}
\definecolor{CodeGreen}{HTML}{2AA1A2}
\definecolor{CodeRed}{HTML}{CB4B16}
\definecolor{CodeYellow}{HTML}{B58900}
\definecolor{CodePurPle}{HTML}{D33682}
\definecolor{CodeGreen2}{HTML}{859900}
\lstset{
    basicstyle=\tt,%字体设置
    numbers=left, %设置行号位置
    numberstyle=\tiny\color{black}, %设置行号大小
    keywordstyle=\color{black}, %设置关键字颜色
    stringstyle=\color{CodeRed}, %设置字符串颜色
    commentstyle=\color{CodeGreen}, %设置注释颜色
    frame=single, %设置边框格式
    escapeinside=`, %逃逸字符(1左面的键),用于显示中文
    %breaklines, %自动折行
    extendedchars=false, %解决代码跨页时,章节标题,页眉等汉字不显示的问题
    xleftmargin=2em,xrightmargin=2em, aboveskip=1em, %设置边距
    tabsize=4, %设置tab空格数
    showspaces=false, %不显示空格
    emph={TRUE,FALSE,NULL,NAN,NA,<-,},emphstyle=\color{CodeBlue2}, %其他高亮}
}


%节标题格式设置
\titleformat{\section}[block]{\large\bfseries}{Exercise \arabic{section}}{1em}{}[]
\titleformat{\subsection}[block]{}{    \arabic{section}.(\alph{subsection})}{1em}{}[]
% \titleformat{\subsubsection}[block]{\normalsize\bfseries}{    \arabic{subsection}-\alph{subsubsection}}{1em}{}[]
% \titleformat{\paragraph}[block]{\small\bfseries}{[\arabic{paragraph}]}{1em}{}[]


% \titleformat{\sectioncommand}[shape]{format}{title-label}{sep}{before-title}[after-title]

% newcommand
\newcommand{\F}{\mathcal{F}}


% 中文字号
% 初号42pt, 小初36pt, 一号26pt, 小一24pt, 二号22pt, 小二18pt, 三号16pt, 小三15pt, 四号14pt, 小四12pt, 五号10.5pt, 小五9pt


\begin{document}

\begin{center}\thispagestyle{plain}

{\LARGE\textbf{STAT 430-2 2025 Winter}}

{\Large\textbf{HW4}}

Tuorui Peng\footnote{TuoruiPeng2028@u.northwestern.edu}
\end{center}

\thispagestyle{myheadings}\markright{Compiled using \LaTeX}
\pagestyle{myheadings}\markright{Tuorui Peng}



% Exercise 6.1.14, 6.1.18, 6.1.19, 6.2.2, 6.2.5, 6.2.8, 6.2.11, 6.2.18, 6.2.22, 6.2.23


\section{Exercise 6.1.14}

\subsection{}
We have
\begin{align*}
    \mathbb{P}_{  }\left( X_{n+1}=1|X_1^n \right) =& \mathbb{E}_{  }\left[ \mathbf{1}_{X_{n+1}=1} | X_1^n \right]  \\
    =& \mathbb{E}_{ \theta  }\left[ \mathbb{E}_{  }\left[ \mathbf{1}_{X_{n+1}=1} | X_1^n, \theta  \right] \right]  \\
    =& \mathbb{E}_{ \theta  }\left[ 1-\theta  \right] \\
    =& 1/2
\end{align*}

\subsection{}

We have 
\begin{align*}
    \mathbb{P}_{  }\left( S_{n+1}=s | S_1^n \right) =& \mathbb{P}_{  }\left( S_{n+1} = s | X_1^n \right)  \\
    =& \mathbb{P}_{  }\left( X_{n+1} = s- \sum_{i=1}^n X_i | X_1^n \right)\\
    =& \mathbb{P}_{  }\left( X_{n+1} = s- S_n | S_n \right)   
\end{align*}
from the previous part. Thus we have $ S_n $ being Markov.


\section{Exercise 6.1.18}
\subsection{}

By property of sets we have $ \Gamma _n \to \Gamma  $ thus $ \mathbf{1}_{\Gamma _n} \xrightarrow[]{\mathrm{a.s.}}  \mathbf{1}_{\Gamma } $, then by L\'evy's upward theorem we have:
\begin{align*}
    \mathbb{P}_{  }\left( \Gamma _n  |X_n \right)  = & \mathbb{E}_{  }\left[ \mathbf{1}_{\Gamma _n} | X_n \right]\\
    =&  \mathbb{E}_{  }\left[ \mathbf{1}_{\Gamma _n} | \F_n \right] \\
    \xrightarrow[L_1]{\mathrm{a.s.}} & \mathbb{E}_{  }\left[ \mathbf{1}_{\Gamma } | \F_\infty \right] \\
    =& \mathbf{1}_{\Gamma }
\end{align*}


\subsection{}

Denote $ K:= \{\omega : X_n(\omega ) \in A_n \, i.o. \} $. Then we have that $ \forall N >0 $, $ \exists n>N $ s.t. $ \mathbb{P}_{  }\left( \Gamma _n \cap K | X_n  \right) \geq \eta >0 $. On the other hand we have
\begin{align*}
    \eta < \mathbb{P}_{  }\left( \Gamma _n \cap K | X_n \right) \xrightarrow[]{\mathrm{a.s.}} \mathbb{P}_{  }\left( \Gamma \cap K | X_{\infty} \right) = \mathbf{1}_{\Gamma \cap K} = 1 = \mathbb{P}_{  }\left( \Gamma \cap K \right) 
\end{align*}
which gives $ \mathbb{P}_{  }\left( K \backslash \Gamma  \right) =0 $. In the above we applied L\'evy's upward theorem to $ \Gamma_n \cap K | X_n $.


\subsection{}

Use $ A_n \equiv A $ and $ B_n \equiv B $ and we have using the precedence:
\begin{align*}
    1 \leq & \mathbb{P}_{  }\left( \{X_n \in A \text{ finitely often} \} \cup (\{X_n \in A \,i.o. \}\backslash \Gamma ) \cup \Gamma   \right)\\
    \leq& \mathbb{P}_{  }\left( \{X_n \in A \text{ finitely often} \} \cup \Gamma   \right) + \mathbb{P}_{  }\left( \{X_n \in A \,i.o. \}\backslash \Gamma   \right)\\
    =& \mathbb{P}_{  }\left( \{X_n \in A \text{ finitely often} \} \cup \Gamma   \right) + 0\\
    =& \mathbb{P}_{  }\left( \{X_n \in A \text{ finitely often} \} \right) + \mathbb{P}_{  }\left( \Gamma   \right)
\end{align*}
where $ \Gamma =\{ X_n\in B \,i.o.\} $ so thus we have proved the claim.


\section{Exercise 6.1.19}
We prove the result for symmetric SRW directly. Denote $ \tau  = \inf\{ k: \omega _k \geq b\} $ and 
\begin{align*}
    h_k(\omega )=& \sum_{i=-\infty}^{+\infty} \mathbf{1}_{\omega _{n-k} = b+i} ,\quad k\in [n]
\end{align*}
and by Strong Markov Property (SMP) we have
\begin{align*}
     \mathbb{P}_{  }\left( \max _{k\leq n} \omega _k \geq b \right) =& \mathbb{E}_{  }\left[ \mathbf{1}_{ \tau \leq n} \right]\\
     =& \mathbb{E}_{  }\left[ \mathbf{1}_{ \tau \leq n}\sum_{\imath = -\infty}^{+\infty}\mathbf{1}_{\omega _n = \imath} \right] \\
     =& \mathbb{E}_{  }\left[ \mathbf{1}_{ \tau \leq n} \mathbb{E}_{  }\left[ \sum_{i=-\infty}^{+\infty} \mathbf{1}_{\theta ^{n\wedge\tau} \omega _{n-n\wedge\tau} = b+i} | \F_{n\wedge\tau} \right]   \right]\\
    =& \mathbb{E}_{  }\left[ \mathbf{1}_{ \tau \leq n} \mathbb{E}_{  }\left[ h_{n\wedge\tau}(\theta ^{n\wedge\tau} \omega ) | \F_{n\wedge\tau} \right]   \right]\\
    \mathop{ = }\limits^{\text{SMP}} & \mathbb{E}_{  }\left[ \mathbf{1}_{ \tau \leq n} \mathbb{E}_{ X_{n\wedge\tau} }\left[ h_{n\wedge\tau}(\omega ) \right]   \right]\\
    =& \mathbb{E}_{  }\left[ \mathbf{1}_{ \tau \leq n} \mathbb{E}_{ X_{n\wedge\tau} }\left[ \mathbf{1}_{\omega _{n-n\wedge\tau} = 0}+ 2\sum_{i=1}^\infty\mathbf{1}_{\omega _{n-n\wedge \tau} = i} \right]   \right]\\
    =& \mathbb{E}_{  }\left[ \mathbf{1}_{ \omega _n = b } \right] + 2\mathbb{E}_{  }\left[  \sum_{i=1}^\infty\mathbf{1}_{\omega _n = b+i} \right]\\
    =& \mathbb{P}_{  }\left(\omega _n = b \right) +  2\mathbb{P}_{  }\left( \omega _n > b \right) 
\end{align*}



\section{Exercise 6.2.2}

\subsection{}

We use $ h_r = \mathbf{1}_{\omega _{n-r} \in B} $ and apply for stopping time $ T_{y,r} $ the SMP to obtain that
\begin{align*}
    \mathbb{P}_{ x }\left( X_n\in B, T_{y,r} \leq n \right)  =& \mathbb{E}_{  }\left[ \mathbf{1}_{X_n\in B} \mathbf{1}_{T_{y,r}\leq n} \right] \\
    =& \mathbb{E}_{  }\left[ \mathbf{1}_{T_{y,r}\leq n} \mathbb{E}_{  }\left[ h_{T_{y,r}}(\theta ^{T_{y,r}}\omega ) | \F_{T_{y,r}} \right] \right] \\
    \mathop{ = }\limits^{\text{SMP}} & \mathbb{E}_{  }\left[ \mathbf{1}_{T_{y,r}\leq n} \mathbb{E}_{ y }\left[ h_{T_{y,r}}(\omega ) \right] \right] \\
    =& \sum_{k=0}^{n-r} \mathbb{P}_{ x }\left( T_{y,r} = n-k \right) \mathbb{P}_{ y }\left( X_{n-k} \in B \right) 
\end{align*}

\subsection{}

Making relabeling $ k \mapsto n-k $ and $ B= \{y\} $ and we have
\begin{align*}
    \mathbb{P}_{ x }\left( X_n=y \right)  =& \sum_{k=r}^n \mathbb{P}_{ x }\left( T_{y,r} = k \right) \mathbb{P}_{ y }\left( X_{n-k} = y \right) 
\end{align*}


\subsection{}

We have
\begin{align*}
     \mathrm{R.H.S.}=& \sum_{n=r}^{l+r}\mathbb{P}_{ y }\left( X_n = y \right) \\
     =& \sum_{n=r}^{l+r}\sum_{k=r}^n \mathbb{P}_{ y }\left( T_{y,r} = k \right) \mathbb{P}_{ y }\left( X_{n-k} = y \right) \\
     =&\sum_{k=r}^{l+r}\mathbb{P}_{ y }\left( T_{y,r}=k \right) \sum_{n=k}^{l+r} \mathbb{P}_{ y }\left( X_{n-k} = y \right) \\
     \leq&\sum_{k=r}^{l+r}\mathbb{P}_{ y }\left( T_{y,r}=k \right) \sum_{n=k}^{l+k} \mathbb{P}_{ y }\left( X_{n-k} = y \right) \\
     =&\sum_{k=r}^{l+r}\mathbb{P}_{ y }\left( T_{y,r}=k \right) \sum_{j=0}^l \mathbb{P}_{ y }\left( X_j = y \right) \\
     \leq&  \sum_{j=0}^l \mathbb{P}_{ y }\left( X_j = y \right)
\end{align*}


\section{Exercise 6.2.5}


\subsection{}

Since $ \mathbb{P}_{ x\not\in C }\left( \tau_C <\infty \right) >0  $, we know that $ \exists N>0 $ and some $ \varepsilon  $ s.t. 
\begin{align*}
    \mathbb{P}_{ x\not\in C }\left( \tau_C <N \right) >\varepsilon 
\end{align*}
and thus we have
\begin{align*}
    \mathbb{P}_{ x\not\in C }\left( \tau_C \geq N \right) \leq 1-\varepsilon  
\end{align*}


Consider applying the SMP to $ h:= \mathbf{1}_{\tau >N} $ we have
\begin{align*}
    \mathbb{E}_{  }\left[ \mathbf{1}_{\tau > (k+1)N} \right] =& \mathbb{E}_{  }\left[ \mathbf{1}_{\tau > kN} \mathbb{E}_{  }\left[ h(\theta ^{kN}\omega ) | \F_{kN} \right] \right] \\  
    =& \mathbb{E}_{  }\left[ \mathbf{1}_{\tau > kN} \mathbb{E}_{ x\not\in C }\left[ h(\omega ) \right] \right] \\
    =& \mathbb{E}_{  }\left[ \mathbf{1}_{\tau > kN} \mathbb{P}_{ x\not\in C }\left( \tau_C \geq N \right) \right] \\
    \leq& \mathbb{E}_{  }\left[ \mathbf{1}_{\tau > kN} (1-\varepsilon ) \right] \\
    \ldots& \leq (1-\varepsilon )^k\mathbb{E}_{ y }\left[ \mathbf{1}_{\tau> N} \right]\\
    \leq& (1-\varepsilon )^k
\end{align*}




\subsection{}

By Borel-Cantelli lemma we have $ \mathbb{P}_{  }\left( \tau_C<\infty \right) =1  $ since
\begin{align*}
    \sum_{k=1}^\infty \mathbb{P}_{ x\not\in C }\left( \tau_C \geq kN \right) <\infty 
\end{align*}
Then we have for any $ x=X_0\not\in C $ that:
\begin{align*}
    g(x)=& \mathbb{P}_{ X_0=x }\left( \tau_A<\tau_B \right)  \\
    =& \mathbb{P}_{ X_0 }\left( X_{\tau_C} \in A \right)\\
    =& \sum_{y\in \mathbb{S}} \mathbb{P}_{ X_0 }\left( X_1 = y \right) \mathbb{P}_{ X_1=y }\left( X_{\tau_C} \in A \right) \\
    =& \sum_{y\in \mathbb{S}} p(x,y)g(y)
\end{align*}
thus $ g(\, \cdot \, ) $ is harmonic on $ \mathbb{S}\backslash C $.


\subsection{}

Note that we have $ X_{n\wedge \tau_C - 1}\not\in C$

By the harmonic proporty we have that 
\begin{align*}
    \mathbb{E}_{  }\left[  g(X_{n\wedge \tau_C + 1 })| \F_n \right] =& \mathbb{E}_{  }\left[  g(X_{n\wedge \tau_C + 1 }) ( \mathbf{1}_{ \tau_C \leq n  } + \mathbf{1}_{\tau _C > n})  | \F_n \right] \\
    =& g(X_{n\wedge \tau_C}) \mathbf{1}_{\tau_C\leq n} + \mathbb{E}_{  }\left[  g(X_{n\wedge \tau_C + 1 }) \mathbf{1}_{\tau_C > n} | \F_n \right] \\
    =&g( X_{n\wedge \tau_C}) \mathbf{1}_{\tau_C\leq n} + \mathbb{E}_{  }\left[ g(\theta ^1 X_{  n\wedge \tau_C  }) \mathbf{1}_{\tau_C > n} | \F_n \right] \\
    \mathop{ = }\limits^{\text{SMP}}& g( X_{n\wedge \tau_C}) \mathbf{1}_{\tau_C\leq n} + \mathbb{E}_{ X_{n\wedge \tau_C} =x\not\in C }\left[ g( X_{  n\wedge \tau_C +1 })  \right]\mathbf{1}_{\tau_C > n} \\
    =& g( X_{n\wedge \tau_C}) \mathbf{1}_{\tau_C\leq n} + g( X_{n\wedge \tau_C}) \mathbf{1}_{\tau_C > n} \\
    =& g( X_{n\wedge \tau_C})
\end{align*}
thus we have $ g(X_{n\wedge \tau_C}) $ is a M.G.

\subsection{}

We have for such M.G. $ g(X_{n\wedge \tau_C}) $ with $ X_0 = x $ that:
\begin{align*}
    g(x)=&\mathbb{E}_{  }\left[ g(X_{0\wedge \tau_C}) \right]\\
    =& \mathbb{E}_{  }\left[ g(X_{n \wedge \tau_C}) \right]\\
    =& \mathbb{E}_{  }\left[ g(X_{\tau_C}) \right] \\
    =& \mathbb{E}_{  }\left[ \mathbf{1}_{X_{\tau_C} \in A} \right]\\
    =& \mathbb{P}_{ x }\left( \tau_A < \tau_B \right) 
\end{align*}
combine with previous results that $ \mathbb{P}_{ x }\left( \tau_A<\tau_B \right)  $ satisfies the condition of bounded harmonic on $ x\not\in C $ we have that it's the only one that satisfies the condition.


\subsection{}

Similar to (c) we prove the following:
\begin{align*}
     \mathbb{E}_{  }\left[ M_{n+1} | \F_n \right] =& \mathbb{E}_{  }\left[ M_{n+1} \mathbf{1}_{\tau_C \leq n} | \F_n \right] + \mathbb{E}_{  }\left[ M_{n+1} \mathbf{1}_{\tau_C > n} | \F_n \right] \\
     =& M_n \mathbf{1}_{\tau_C \leq n} + \mathbb{E}_{  }\left[ M_{n+1} \mathbf{1}_{\tau_C > n} | \F_n \right] \\ 
     =& M_n \mathbf{1}_{\tau_C \leq n} + (n+1 + \mathbb{E}_{  }\left[ f(X_{n + 1}) | \F_n \right])\mathbf{1}_{\tau_C > n}\\ 
     =& M_n \mathbf{1}_{\tau_C \leq n} + (n+1 + \mathbb{E}_{ X_n=x\not\in C }\left[ f(X_{n+1}) \right] )\mathbf{1}_{\tau_C > n}\\
     =& M_n \mathbf{1}_{\tau_C \leq n} + (n+1 + f(X_n)-1 )\mathbf{1}_{\tau_C > n}\\
     =& M_n
\end{align*}
thus $ M_n $ is a M.G. On the other hand we notice that
\begin{align*}
    \mathbb{E}_{  }\left[ M_n \right] =& \mathbb{E}_{  }\left[ n\wedge \tau_C + f(X_{n\wedge \tau_C}) \right] \\
    \leq& \mathbb{E}_{  }\left[ \tau_C \right] + \sum_{x\in \mathbb{S}}f(x) \mathbb{P}_{  }\left( X_{n\wedge \tau_C} = x \right) \\ 
    \leq& \mathbb{E}_{  }\left[ \tau_C \right] + \sum_{x\in \mathbb{S}\backslash C}f(x) + \sum_{x\in C} 0 <\infty
\end{align*}
then by DCT we have that
\begin{align*}
    \mathbb{E}_{ x }\left[ \tau_C \right] =& \mathbb{E}_{  }\left[ \lim_{n\to\infty} n\wedge\tau_C + f(X_{n\wedge \tau_C}) \right] \\
    =& \lim_{n\to\infty} \mathbb{E}_{  }\left[ n\wedge\tau_C + f(X_{n\wedge \tau_C}) \right] \\
    =&   \mathbb{E}_{ x }\left[ 0 + f(X_{0\wedge \tau_C}) \right] \\
    =& f(x)
\end{align*}


\section{Exercise 6.2.8}

\subsection{}
We have that
\begin{align*}
    \rho _{xz}=& \mathbb{E}_{ x }\left[ \mathbf{1}_{T_z<\infty} \right]\\
    \geq& \mathbb{E}_{ x }\left[ \mathbf{1}_{T_z <\infty}\mathbf{1}_{T_y<T_z} \right]   \\
    =& \mathbb{E}_{ x }\left[ \mathbf{1}_{T_y < T_z}\mathbb{E}_{  }\left[ \theta ^{T_y}\mathbf{1}_{T_z<\infty } | \F_{T_y} \right]  \right]   \\
    \mathop{ = }\limits^{\text{SMP}}& \mathbb{E}_{ x }\left[ \mathbf{1}_{T_y < T_z}\mathbb{E}_{ y }\left[ \mathbf{1}_{T_z<\infty } \right]  \right]   \\ 
    =&\rho _{xy}\rho _{yz} > 0
\end{align*}

\subsection{}

We have
\begin{itemize}[topsep=2pt,itemsep=0pt]
    \item $ x\leftrightarrow x $ by convention;
    \item $ x\leftrightarrow y $ implies $ y\leftrightarrow x $ by symmetry of the condition $ \rho _{xy}>0 $, $ \rho _{yx}>0 $;
    \item $ x\leftrightarrow y $ and $ y\leftrightarrow z $ implies $ x\leftrightarrow z $ by the previous part.
\end{itemize}

\subsection{}

\begin{itemize}[topsep=2pt,itemsep=0pt]
    \item If $ C_1 $ leads to $ C_2 $ and $ C_2 $ leads to $ C_3 $, then there exists $ x,y\in C_2 $ s.t. $ x\leftrightarrow y $ and thus for any $ z\in C_1 $ and $ w\in C_3 $ we have $ z\to  x \to y \to w $, thus $ z\leftrightarrow w $, i.e. $ C_1 $ leads to $ C_3 $.
    \item If $ C_1 $ leads to $ C_2 $ and $ C_2 $ leads to $ C_1 $, then for any $ x\in C_1 $ and $ y\in C_2 $ we have $ x\to y$ and $ y\to x $, which turns out to be $ x\leftrightarrow y $, which is a contradiction, so if $ C_1 $ leads to $ C_2 $, then $ C_2 $ cannot lead to $ C_1 $.
\end{itemize}

\section{Exercise 6.2.11}


\subsection{}
We have $ \forall k $ and any bounded function $ g $ that
\begin{align*}
    \mathbb{E}_{ y }\left[ g(r_k,X_{R_{k-1}},\ldots X_{R_{k}-1}) | \F_{R_{k-1}} \right]=& \mathbb{E}_{ y }\left[ \theta ^{R_{k-1}}g(\omega ) | \F_{R_{k-1}} \right] \\
    \mathop{ = }\limits^{\text{SMP}} & \mathbb{E}_{ y }\left[ g(\omega ) \right] \\
    =& \mathbb{E}_{ y }\left[ g(r_0,X_0,\ldots X_{R_{0}-1}) \right]
\end{align*}
and notice that $ \F_{k-1}\supseteqq \sigma (r_1^{k},X_1^{R_{k}-1}) $ we have that $ (r_k,X_{R_{k-1}},\ldots X_{R_{k}-1}) $ are independent and identically distributed as $ (r_0,X_0,\ldots X_{R_{0}-1}) $.

\subsection{}

Similarly we have
\begin{align*}
    \mathbb{E}_{ \nu  }\left[ g(r_k,X_{R_{k-1}},\ldots X_{R_{k}-1}) | \F_{R_{k-1}} \right]
    =&\mathbb{E}_{ \nu  }\left[ \mathbb{E}_{  }\left[ \theta ^{R_{k-1}} g(\omega ) | \F_{R_{k-1}} \right] |F_{R_1} \right]\\
    =&\mathbb{E}_{ \nu  }\left[ \mathbb{E}_{ y }\left[ \theta ^{R_1}g(\omega ) \right] |\F_{R_1} \right]  \\
    =&\mathbb{E}_{ y }\left[ g(r_2,X_{R_{1}},\ldots X_{R_{2}-1}) \right]
\end{align*}
thus proved that $ (r_k,X_{R_{k-1}},\ldots X_{R_{k}-1}) $ are independent and identically distributed as $ (r_1,X_{R_{1}},\ldots X_{R_{2}-1}) $, $ \quad k\geq 2 $.

\section{Exercise 6.2.18}

Denote the initial distribution as $ X_0\sim \nu  $.

\subsection{}
Take $ \nu =\delta _0 $, then we have 
\begin{align*}
 0 =& \mathbb{E}_{ \nu  }\left[ X_0 \right] =0\\
 =& \mathbb{E}_{ \nu  }\left[ X_1 \right] \\
 =& \sum_{i=0}^N \mathbb{P}_{  }\left( X_0=0 \right)p(0,i)\\
  \Rightarrow & p(0,i) =0, \quad i\neq 0 
\end{align*}
the above argument is similar for $ \nu =\delta _N $. Together we have $ p(0,0)=p(N,N)=1 $.

\subsection{}

We have for any $ x\not\in \{0,N\} $:
\begin{align*}
    1-\rho _{xx}=& \mathbb{P}_{ x }\left( \tau_x=\infty \right)  \\
    \geq& \mathbb{E}_{ x }\left[ \mathbf{1}_{\tau_x=\infty} \mathbf{1}_{T_{\{0,N\}}<\infty} \right]\\
    =& \mathbb{E}_{ x }\left[ \mathbf{1}_{\tau_{\{0,N\}}<\infty} \mathbb{E}_{  }\left[ \theta ^{\tau_{\{0,N\}}}\mathbf{1}_{T_x=\infty} | \F_{\tau_{\{0,N\}}} \right]  \right]\\
    \mathop{ = }\limits^{\text{SMP}}=& \mathbb{E}_{ x }\left[ \mathbf{1}_{\tau_{\{0,N\}}<\infty} \mathbb{E}_{ X_{\tau_{\{0,N\}}} }\left[ \mathbf{1}_{\tau_x=\infty}  \right]  \right]\\   
    =&\mathbb{E}_{ x }\left[ \mathbf{1}_{\tau_{\{0,N\}}<\infty} \right] >0
\end{align*}
thus proved that any $  x\not\in \{0,N\} $ is trancient.

Then consider for any $ x $ and choose $ \nu = \delta _x $. We have
\begin{align*}
    \mathbb{E}_{ x }\left[ X_{\tau_{\{0,N\}}} \right] =& \mathbb{E}_{ x }\left[ X_0 \right] = x 
\end{align*}
on the other hand we know that
\begin{align*}
     \mathbb{E}_{ x }\left[ X_{\tau_{\{0,N\}}} \right] =& \mathbb{P}_{ x }\left( \tau_0<\tau_N \right) \times 0 + \mathbb{P}_{ x }\left( \tau_0>\tau_N \right) \times N = N\mathbb{P}_{ x }\left( \tau_0>\tau_N \right) 
\end{align*}
together we have $ \mathbb{P}_{ x }\left( \tau_0>\tau_N \right) = x/N $.


\subsection{}

We have for $ S_n $ denoting the symmeric SRW that $ \{S_n^2-n\} $ is a M.G. by checking the following:
\begin{align*}
    \mathbb{E}_{  }\left[ S^2_{n+1} - (n+1)|\F_n \right] =& \mathbb{E}_{  }\left[ S_n^2 + \xi _{n+1}^2 + 2S_n\xi _{n+1} - (n+1) |S_n \right] = S_n^2 -n
\end{align*}
and thus we have
\begin{align*}
    \mathbb{E}_{ x }\left[ S^2_{ \tau_{\{0,N\}}} - \tau_{\{0,N\}}\right]  =& \mathbb{E}_{ x }\left[ S^2_{0\wedge \tau_{\{0,N\}}} - 0\wedge \tau_{\{0,N\}} \right] =x^2
\end{align*}
on the other hand we have
\begin{align*}
    \mathbb{E}_{ x }\left[ S^2_{ \tau_{\{0,N\}}} - \tau_{\{0,N\}}\right]  =& \dfrac{ x }{ N }(N^2 - \mathbb{E}_{ x }\left[  \tau_{\{0,N\}} \right]) + \dfrac{ N-x }{ N }(0 - \mathbb{E}_{ x }\left[  \tau_{\{0,N\}} \right]) 
\end{align*}
together we have $ \mathbb{E}_{ x }\left[  \tau_{\{0,N\}} \right] = x(N-x) $.



\section{6.2.22}



We construct $ h(x)=\mathbf{1}_{x\not\in [-r,r]} $ and we have that $ G_{1/2} = [-r,r]\cap \mathbb{Z} $. We verify the super-harmonic as follows: for any $ x\in \mathbb{Z}\backslash G_r $ we have
\begin{align*}
    1=& h(x)\\
    \geq& \sum_{m=-r}^{r}\mathbb{P}_{  }\left( \xi _i = m \right) \\
    =&\sum_{y=x-r}^{x+r}p(x,y)\\
    \geq& \sum_{y=x-r}^{x+r}p(x,y)h(y)
\end{align*}
since we have bounded difference of the chain. Thus by prop 6.2.21 we know that $ S_n $ is a recurrent chain.



\section{6.2.23}


\begin{itemize}[topsep=2pt,itemsep=0pt]
    \item If constant function, WLOG $ f(x)\equiv 1 $, is super-harmonic, then we use the previous argument that choose $ h(x)=\mathbf{1}_{x\in \mathbb{S}\backslash F} $ where $ F $ is some finite set in $ \mathbb{S} $, in this way we have $ h $ being super-harmonic on $ \mathbb{S}\backslash F $ and thus by prop 6.2.21 we have that $ S_n $ is recurrent.
    \item If $ S_n $ is recurrent, we consider the stopping time $ \tau_u:= \inf_n\{S_n = u\} $ where $ u $ is some state in $ \mathbb{S} $ and by recurrence we have $ \mathbb{P}_{ x }\left( \tau_u <\infty \right) =1 $ for any $ x\in \mathbb{S} $, then we have that a super-harmonic function $ f $ should be s.t. $ f(S_n) $ is a super-martingale, using the condition we see that $ \forall x\in \mathbb{S} $:
    \begin{align*}
        f(x)=&\mathbb{E}_{ x }\left[ f(S_0) \right]\\
        \geq& \mathbb{E}_{ u }\left[ f(S_{\tau_u}) \right]\\
        =& f(u)
    \end{align*}
    which means that $ f $ can only be a constant function.
\end{itemize}

    













    

    




































\end{document}
